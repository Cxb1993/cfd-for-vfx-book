%%%%%%%%%%%%%%%%%%%%%%preface.tex%%%%%%%%%%%%%%%%%%%%%%%%%%%%%%%%%%%%%%%%%
% sample preface
%
% Use this file as a template for your own input.
%
%%%%%%%%%%%%%%%%%%%%%%%% Springer %%%%%%%%%%%%%%%%%%%%%%%%%%

\preface

%% Please write your preface here
%Use the template \emph{preface.tex} together with the Springer document class SVMono (monograph-type books) or SVMult (edited books) to style your preface in the Springer layout.

%A preface\index{preface}5 is a book's preliminary statement, usually written by the \textit{author or editor} of a work, which states its origin, scope, purpose, plan, and intended audience, and which sometimes includes afterthoughts and acknowledgments of assistance. 

%When written by a person other than the author, it is called a foreword. The preface or foreword is distinct from the introduction, which deals with the subject of the work.


how it came to be

gpu hpc 

its right when it looks right

while engineering wants to receive accurate results, in graphics and feature film the credo is "If it looks rights, it is right". Focus is on many look iterations, giving creative input and achieve realtime capabilites as far as possible. 

GPU implementations will be discussed and shown on CUDA hardware.



The field of Computational Fluid Dynamics (CFD) is widely utilized in science, engineering and has found its way into game engines and visual effects for feature film. This book shall give a detailed introduction to the topic of CFD for readers in the visual effects and computer graphics industry. Furthermore it will shed light on the differences between mesh-based, mesh-free and hybrid methods to numerically model fluid dynamic problems. This book in taking the reader on a journey from physical aspects over the mathematical aspects to the numerical aspects and finally implementation of a CFD solver. example paper


Purpose This book should give a TD working in the field of VFX a bottom up solid introduction into CFD, so he is able to implement own CFD solvers when following the most recent research.
limitations
scope


%"A preface or foreword deals with the genesis, purpose, limitations, and scope of the book an
%Customarily \textit{acknowledgments} are included as last part of the preface.

\vspace{\baselineskip}
\begin{flushright}\noindent
Hamburg,\hfill {\it Jathavan Sriram}\\
December 2014\hfill\\
\end{flushright}


